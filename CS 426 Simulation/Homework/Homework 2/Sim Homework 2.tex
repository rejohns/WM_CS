\documentclass[11pt]{article} % 
\usepackage[pdftex]{graphicx}
\usepackage{fullpage}
\usepackage{graphicx}
\usepackage{graphics}
\usepackage{psfrag}
\usepackage{pgf}
\usepackage{color}
\usepackage{tikz}
\usetikzlibrary{arrows,automata}
\usepackage[latin1]{inputenc}
\usepackage{amsthm}
\usepackage{amsmath,amssymb}
\usepackage{enumerate}
\setlength{\textwidth}{6.5in}
\setlength{\textheight}{9in}
\newcommand{\cP}{\mathcal{P}}
\newcommand{\N}{\mathbb{N}}
\newcommand{\Z}{\mathbb{Z}}
\newcommand{\R}{\mathbb{R}}
\newcommand{\Q}{\mathbb{Q}}
\newcommand{\C}{\mathbb{C}}
\newcommand{\tab}{\;\;\;\;\;}
\newcommand{\inv}{^{-1}}

\begin{document}

\hfill Robert Johns

\hfill September 23, 2013

\begin{center} {\Large CSCI 426: Simulation}\\{\large Homework 2}\end{center}

\begin{enumerate}

%1
\item [\bf{2.1.9}]

\begin{enumerate}

%1a
\item Verify that the list of five full-period multipliers in Example 2.1.6 is correct.

{\bf Solution:} The following program was used to verify the list:

{\small{\begin{verbatim}
int main(void) {
        	long p = 1;
        	long a = 447489615;
        	long x = a;
        	long m = pow(2,31) - 1;
        	while (x != 1) {
        	        	p++;
                	x = (a*x)%m;
        	}
        	if (p == m - 1) {
                	printf("yea\n");
        	}
        	else {
                	printf("nay\n");
        	}
        	return 0;
}
\end{verbatim}}}

%1b
\item What are the next five elements in the list?

{\bf Solution:} The following code was used to find the next five:

{\small{\begin{verbatim}
int gcd (long a, long b) {
        int c;
        	while ( a != 0 ) {
                	c = a; a = b%a;  b = c;
        	}
        	return b;
}
int main(void) {
        	long i = 1;
        	long a = 7;
        	long x = a;
        	long count = 0;
        	long m = pow(2,31) - 1;
        	while(x != 1 && count < 10) {
                	if (gcd(i, m -1) == 1) {
                        	printf("yea: x = %d, i = %d\n", x, i);
                        	count++;
                	}
                	i++;
                	x = (a*x)%m;
        	}
        	return 0;
}
\end{verbatim}}}

The next five values are, where $x = 7^i\mod 2^{31}-1$:

$\tab x = 680742115, i = 23$

$\tab x = 1144108930, i = 25$

$\tab x = 373956417, i = 29$

$\tab x = 655382362, i = 37$

$\tab x = 1615021558, i = 41$

\end{enumerate}

%2
\item[\bf{2.2.13}] if $m=2^{31}-1$, compute the $x\in\chi_m$ for which $7^x\mod m = 48271$.

{\bf Solution:} The following code, a slightly modified version of the program for {\bf 2.1.9(b)} was used to find the solution:

{\small{\begin{verbatim}
int gcd (long a, long b) {
        	int c;
        while ( a != 0 ) {
                c = a; a = b%a;  b = c;
        }
        return b;
}
int main(void) {
        long x = 1;
        long a = 7;
        long y = a;
        long m = pow(2,31) - 1;
        while(y != 1) {
                if (gcd(i, m -1) == 1 && y == 48271) {
                        printf("x = %d\n", x);
                        return 0;
                	}
                	x++;
                	y = (a*y)%m;
        }
        return 0;
}
\end{verbatim}}}

The program yielded the result: $x = 1116395447$

%3
\item[\bf{2.3.3}] A fair coin is tossed once.  If it comes up heads, a fair die is rolled, and you are paid the number showing in dollars.  If it comes up tails, two fair dice are rolled and you are paid the sum of the two numbers showing in dollars.  Let $X$ be the amount won.  Enumerate all possible values of $X$and use Monte Carlo simulation to estimate the probability of each.

{\bf Solution:} The possible values, and the axiomatic probabilities of each, are:

\begin{enumerate}

\item[1:] $\frac{1}{2}*\frac{1}{6} = .083$

\item[2:] $\frac{1}{2}*\frac{1}{6} + \frac{1}{2}*\frac{1}{36} = .097$

\item[3:] $\frac{1}{2}*\frac{1}{6} + \frac{1}{2}*\frac{2}{36} = .111$

\item[4:] $\frac{1}{2}*\frac{1}{6} + \frac{1}{2}*\frac{3}{36} = .125$

\item[5:] $\frac{1}{2}*\frac{1}{6} + \frac{1}{2}*\frac{4}{36} = .138$

\item[6:] $\frac{1}{2}*\frac{1}{6} + \frac{1}{2}*\frac{5}{36} = .152$

\item[7:] $\frac{1}{2}*\frac{6}{36} = .083$

\item[8:] $\frac{1}{2}*\frac{5}{36} = .069$

\item[9:] $\frac{1}{2}*\frac{4}{36} = .055$

\item[10:] $\frac{1}{2}*\frac{3}{36} = .041$

\item[11:] $\frac{1}{2}*\frac{2}{36} = .027$

\item[12:] $\frac{1}{2}*\frac{1}{36} = .013$

\end{enumerate}

The following program was used as a Monte Carlo simulation for the scenario (\texttt{Random()} uses the rng.c file provided):

{\small{\begin{verbatim}
int main(void) {
        	int pos[] = {0,0,0,0,0,0,0,0,0,0,0,0};
        double count = 0;
        while(count < 100000000) {
                double tr1 = Random() * 2;
                int r1 = floor(tr1);
                if (r1 == 0) {
                        int tr2 = Random() * 6;
                        int r2 = floor(tr2);
                        pos[r2]++;
                }
                else {
                        double tr3 = Random() * 6;
                        int r3 = floor(tr3) + 1;
                        double tr4 = Random() * 6;
                        int r4 = floor(tr4) + 1;
                        int sum = r3 + r4;
                        pos[sum-1]++;
                }
                count++;
        }
        int j;
        for(j = 0; j < 12; j++) {
                double prob = pos[j]/count;
                printf("Probability of %d: %.5f\n", j+1, prob);
        }
        return 0;
}	
\end{verbatim}}}

The program yielded the following results, which support the axiomatic probabilities:

$\tab$Probability of 1: 0.08331
$\tab$Probability of 2: 0.09727
$\tab$Probability of 3: 0.11111

$\tab$Probability of 4: 0.12506
$\tab$Probability of 5: 0.13888
$\tab$Probability of 6: 0.15273

$\tab$Probability of 7: 0.08333
$\tab$Probability of 8: 0.06943
$\tab$Probability of 9: 0.05554

$\tab$Probability of 10: 0.04168
$\tab$Probability of 11: 0.02776
$\tab$Probability of 12: 0.01390

\end{enumerate}

\end{document}